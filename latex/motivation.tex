\section{Motivation}
\label{sec:motivation}
	
Shape optimization is inherently an infinite-dimensional problem,
for which tractable formulations are of particular interest.
One approach is to optimize directly for the continuous quantity of interest, a function, which is often done iteratively.
Starting with an initial guess, partial differential equations are evaluated over the domain of interest,
and then the shape is perturbed until some optimality condition is reached.

Another approach is to discretize the space, which has other difficulties.
[See \textit{Pros and Cons of Airfoil Optimization} by Mark Drela].
However, this method can be used to great effect when the thing being designed is
uncontinuous, such as things meant to be additively manufactured.

These is very little literature, if any, on rocket conceptual design,
for several reasons. (1) The physics governing internal reactive flow is extremely complex,
and there are no open-source tools that don't require domain knowledge and
can efficiently simulate such a scenario.
(2) There are few entities that design and build solid rocket motors, and since they have
large amounts of resources and time and little competition there has been no impetus
improve solid rocket design methods. (3) Solid rocket motors are relatively limited
in capability because their burn rate is almost completely uncontrollable other
than by properly designing the internal geometry of the rocket or potentially actuating the
throat of the nozzle. For this reason liquid propellant rockets have been preferred for
many applications. (4) Rocket design is often proprietary due to arms regulations.

As a result, it is worthwhile from both a theoretical and practical standpoint
to come up with a tractable, physics-based framework to perform the conceptual design of solid rocket
motors that does not rely on prior experience and vast amounts of time and capital resources.
