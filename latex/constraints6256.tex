   \section{Constraint definition}
   \label{sec:generalconstraints}

    This section elaborates on the general form of the constraints of the problem.
    The constraints are imposed at $n_x+1$ cross-sections to determine the polynomials describing
    the square of the radius at different cross-sectional slices of the rocket.
    The surface geometry in between the sections will be described through interpolation.

    \begin{itemize}
        \item \textbf{The integral difference between the cross-sectional areas of the shape must be equal
        to the fuel burnt in a given time step, as determined by the first-stage problem.} In other words
        for the $i$th slice of the rocket:
        \begin{equation}
            \int\limits_{-\pi}^{\pi} r_{i,t}^2 d\theta = A_{i,t}
        \end{equation}
        We can easily represent this constraint by taking the integral of the polynomial $r_{i,t}^2(\theta)$
        which makes this a linear constraint on the coefficients of the polynomial.
        \item \textbf{The arc length of the polynomial must be equal to the arc length $l_b$ as determined
        by the first-stage problem.} The arc length of an arbitrary polynomial curve has no closed-form solution, so we
        have to use the integral form below:
        \begin{equation}
            C_{i,t} = \int\limits_{-\pi}^{\pi} \sqrt{r_{i,t}^2 + \Big(\frac{dr_{i,t}}{d\theta}\Big)^2} d\theta
        \end{equation}
        This integral form unfortunately has no \gls{sos} representation. This means
        that I will implement an outer loop problem to make sure that the polynomials
        have the appropriate arc lengths.
        \item \textbf{The shapes must never intersect.}
        Since the flame front must strictly regress as a result of burning,
        the polynomials describing the shapes must not intersect.
        This is a sum-of-squares constraint on the difference of the univariate polynomials, where
        $r^2_{i,t}(\theta) - r^2_{i,t-1}(\theta) \rm{~is~SOS~over~} \theta \in [-\pi, \pi]$.
        \item\textbf{The shapes must not intersect the unit circle.}
        Since fuel can only be contained inside the casing of the rocket,
        we set an upper bound on the polynomials. This can be achieved by making sure
        the polynomials don't intersect with the unit circle. This means that
        $R^2 - r^2_{i,n_t}(\theta) \rm{~is~SOS~over~} \theta \in [-\pi, \pi]$, where
        $R$ is the outer radius of the \gls{srm}.
        \item \textbf{Regularity conditions on the shape of the polynomials.}
        The first regularity condition is that the polynomials are periodic over $[-\pi, \pi]$.
        I tried imposing conditions on the first and second derivative of the polynomial over the boundary,
        but this ended up making the problem less well-posed (solver SeDuMi kept running
        into numerical problems, the polynomials were not remotely close to satisfying constraints).
    \end{itemize}

