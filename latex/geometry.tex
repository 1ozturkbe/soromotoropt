\section{Geometry in polar coordinates}

Frankly, I underestimated the difficult algebra required to approximate
arc lengths and areas in polar coordinates. So I will detail it here
before I describe the constraints in my problem. Assuming that we
know the radius of our shape $r(\theta)$ for $\theta=[-\pi, \pi]$,
the circumference and area of the shape is described as follows.

\begin{align}
    L & = \int\limits_{\pi}^{\pi} \sqrt{r^2 + \Big(\frac{dr}{d\theta}\Big)^2} d\theta \\
    A & = \int\limits_{\pi}^{\pi} r^2 d\theta
\end{align}

This poses some problems, since the integral of the square root of a polynomial
is not a polynomial that is amenable to \gls{sos} representation, since the
constraints need to be linear over the coefficients of the polynomial.
I tried a number of different approximations for both the area and the circumference, but settled
on describing $r^2_{i,t}(\theta)$ as a \gls{sos} polynomial so I could
optimize the polynomials over the cross-sectional areas exactly. I resolved to
get the arc lengths by perturbing the output polynomials of my \gls{sos} problem
in a different optimization step.
