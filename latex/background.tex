\section{Brief background on solid rocket motors}

A \gls{srm} is extremely simple. Think a candle with a casing on it to contain the pressure and hot gas from burning material,
and a nozzle at the end to make the high-pressure flow go supersonic and generate thrust.
There are a few differences between \gls{srm}s and candles.
First is the fact that solid rocket motor fuel contains propellant and oxidizer (wax and oxygen),
whereas a candle only contains fuel and relies on an external source of oxygen (eg. the air around it).
As a result, \gls{srm}s can even be lit underwater or in a vacuum;
they are extremely robust. The second difference is the geometry.
A candle is akin to an end-burning rocket; the wax only burns at the candle's melty end.
A \gls{srm} usually has some geometry down the bore of the rocket which runs all the way down its length for greater surface area.
(An end-burning rocket is almost completely uninteresting since there is not a significant aspect of geometry to design.)
The third is the presence of the nozzle. Without it, the flow would stay sonic when exiting the bore and
not generate thrust effectively. The disadvantage of solid rocket motors is that
they are essentially a runaway chemical reaction, an explosive that is confined in a casing,
whose rate is controlled by the area of the throat of the nozzle.
Once lit, they cannot be stopped or controlled, like liquid rockets or jet engines
can be by modulating the amount of propellant and oxidizer.
Therefore the design of the geometric pattern in a \gls{srm}'s bore is critical for a proper burn rate.
