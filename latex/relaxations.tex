\section{Relaxations and post-processing}
\label{sec:relaxations}

Now that we have a feasible solution to the finite volume problem
with relaxed conservation laws, we need to determine how to best
accommodate these relaxations to map propellant into the interior of
the solid rocket motor. A key point to remember is that all of the relaxations are \emph{conservative},
i.e. mass, momentum and energy can only be lost, not created. Ideally,
we have a solution that does not lose in any of these quantities since it will
result in the minimum cost solution possible. However, the fact that thrust is an \emph{input}
to the model and not a variable makes this almost surely impossible.

There are eight types of equality constraints that are relaxed in the formulation:
\begin{itemize}
    \item Nozzle total pressure
    \item Nozzle total temperature
    \item Fuel mass conservation
    \item Total temperature within SRM section
    \item Burn rate within SRM section
    \item Mass conservation within SRM section
    \item Momentum conservation within SRM section
    \item Energy conservation within SRM section
\end{itemize}

Empirically, it was found that the first three constraints are always tight, which
somewhat simplifies the analysis. Furthermore, we can ignore the relaxations of
total temperature, since the effect of this is properly captured in the energy conservation constraint.
So the ones that we will be considering in the following sections are burn rate, and
mass, momentum and energy conservation laws.

\subsection{Types of materials within rocket motor}

We will assume that the propellant is made up of three different materials

\begin{itemize}
    \item \textbf{Propellant}: This is the chemical that
    provides the chemical potential energy and mass to the reaction.
    \item \textbf{Accelerant}: This chemical contributes mass but does not contribute
    to the heat of combustion of the reaction, but increases the rate at which
    propellant burns.
    \item \textbf{Filler}: This chemical contributes mass but does not contribute
    to the heat of combustion of the reaction. It slows down the rate of reaction by
    its presence.
\end{itemize}

\subsection{Mass flow relaxation}

The loss of mass through the burn can be thought of as the presence of voids in the
propellant. We define $p_f$ as the porosity of the section, which describes this.
It is found that mass conservation laws are often tight to a thousandth of the constraint,
so porosity values of the propellant area expected to be low.

\subsection{Burn rate relaxation}

\begin{equation}
    1 + r_b = \Big(1+\frac{\beta_a}{\beta_p}\Big) (1-\beta_f + p_f),~\beta_p + \beta_a + \beta_b = 1
\end{equation}



