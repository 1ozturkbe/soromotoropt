\section{Approach}

To tackle this problem, we will first make a series of approximations to make it more tractable.

\subsection{Assumptions}
\label{sec:assumptions}

\begin{itemize}
	\item \textbf{The internal flow is steady and isentropic:} This will allow us to make approximations for the static pressure inside the rocket, which is related to the burn rate.
	\item \textbf{Pressure dependence on the burn rate is defined by the Saint-Robert's law:} $r = a P_c^n$
	\item \textbf{Constant radial burn rate throughout the chamber:} This allows us to treat the problem as a 2D problem. This approximation is justifiable since there is a relative pressure drop in the bore as the fluid accelerates through it. The greater rate of erosive burning near the exit port is offset by the lower pressure near the port.
	\item \textbf{The grain is not temperature sensitive:} This is definitely not a correct assumption, but since this is a proof of concept we can assume this in the beginning.
\end{itemize}

\subsection{First stage problem}

Starting with $n_{x}$ radial slices of the rocket evolving over $n_t$ time steps,
we devise a low-fidelity method to determine how the aggregate geometric properties of the rocket
evolve over time, such as the cross-sectional area and arc length of the fuel.

%\begin{tikzpicture}[scale=2]
%	\foreach \pos [count=\Ind] in {{0,0,1},{1,0,1},{0,0,0},{1,0,0},{0,1,1},{1,1,1},{0,1,0},{1,1,0}}{
%	\node[circle,inner sep=1pt,fill=black,label=\ifnum\Ind<7 below\else above\fi:\Ind](p\Ind) at (\pos){};
%	}
%	\draw (p1)--(p2)--(p3)--(p6)--(p12)--(p11)--(p10)--(p7)edge(p1)--(p8)edge(p11)edge(p2)--(p9)edge(p12)--(p3);
%	\draw[dotted] (p4)edge(p1)edge(p10)--(p5)edge(p2)edge(p11)--(p6);
%\end{tikzpicture}

%\tdplotsetmaincoords{60}{30}
%\begin{tikzpicture}[tdplot_main_coords,>=latex,line join=bevel,font=\sffamily,
%bullet/.style={circle,fill,inner sep=1pt},scale=4]
%  \foreach \X in {0,1}
%  {\foreach \A [evaluate=\A as \Y using {(cos(3.14159*\A/2)+0.5)}] in {1,8}
%  {\foreach \B [evaluate=\B as \Z using {(sin(3.14159*\B/2)+0.5)}] in {1,8}
%  {
%  	\path (\X,\Y,\Z) node[bullet, label=below:\A], () {};
%}}}
%%  \draw[dotted] (1)  -- (4) -- (6)
%%  (4) -- (10)  (2) -- (5) -- (11);
%\end{tikzpicture}

%		\section{Preliminary Optimization Problem}
%	
%	To test whether or not the method of vector translations can be solved
%	in a mixed integer convex form, we formulate the following problem 
%	
%	\subsection{Parameters}
%	\begin{itemize}
%	\item \textbf{$n_{ctrl}$:} number of control points
%	\item \textbf{$n_t$:} number of time steps
%	\item \textbf{$r$:} radius of the rocket
%	\item \textbf{$R$:} regression rate of fuel
%	\item \textbf{$C_{profile}$:} burn surface length desired (vector of $n_t$)
%	\end{itemize}
		
