% --------------------------------------------------------------
% This is all preamble stuff that you don't have to worry about.
% Head down to where it says "Start here"
% --------------------------------------------------------------

\documentclass[12pt]{article}

\usepackage[margin=1in]{geometry} 
\usepackage{amsmath,amsthm,amssymb}
\usepackage{comment}
\usepackage{graphicx}

\newcommand{\N}{\mathbb{N}}
\newcommand{\Z}{\mathbb{Z}}

\newenvironment{theorem}[2][Theorem]{\begin{trivlist}
		\item[\hskip \labelsep {\bfseries #1}\hskip \labelsep {\bfseries #2.}]}{\end{trivlist}}
\newenvironment{lemma}[2][Lemma]{\begin{trivlist}
		\item[\hskip \labelsep {\bfseries #1}\hskip \labelsep {\bfseries #2.}]}{\end{trivlist}}
\newenvironment{exercise}[2][Exercise]{\begin{trivlist}
		\item[\hskip \labelsep {\bfseries #1}\hskip \labelsep {\bfseries #2.}]}{\end{trivlist}}
\newenvironment{reflection}[2][Reflection]{\begin{trivlist}
		\item[\hskip \labelsep {\bfseries #1}\hskip \labelsep {\bfseries #2.}]}{\end{trivlist}}
\newenvironment{proposition}[2][Proposition]{\begin{trivlist}
		\item[\hskip \labelsep {\bfseries #1}\hskip \labelsep {\bfseries #2.}]}{\end{trivlist}}
\newenvironment{corollary}[2][Corollary]{\begin{trivlist}
		\item[\hskip \labelsep {\bfseries #1}\hskip \labelsep {\bfseries #2.}]}{\end{trivlist}}

\begin{document}
	
	% --------------------------------------------------------------
	%                         Start here
	% --------------------------------------------------------------
	
	%\renewcommand{\qedsymbol}{\filledbox}
	
	\title{Solid Rocket Motor Optimization for Additive Manufacturing}
	\author{Berk Ozturk}
	
	\maketitle
	
	\section{Motivation}
	
	\section{Concept}
	
	Starting with $n_{ctrl}$ control points evolving over $n_t$ time steps, we can devise a method to appropriately mesh the 2D slice of the rocket. 
	The hope is that the problem can be posed as a difference-of-convex optimization problem, and especially a mixed integer convex problem through piecewise linearization and the use of binary variables. 
	The pipe dream is to be able to perform this optimization in 3D with good first-order methods to accommodate for erosive burning. 
	\subsection{Approximations}
	\begin{itemize}
		\item \textbf{Pressure dependence on the burn rate is defined by the Saint-Robert's law:} $r = a P_c^n$
		\item \textbf{Constant radial burn rate throughout the chamber:} This allows us to treat the problem as a 2D problem. This approximation is justifiable since there is a relative pressure drop in the bore as the fluid accelerates through it. The greater rate of erosive burning near the port is offset by the lower pressure near the port. 
		\item 
	\end{itemize}
	
	
	\section{Preliminary Optimization Problem}
	
	\subsection{Parameters}
	\begin{itemize}
	\item \textbf{$n_{ctrl}$:} number of control points
	\item \textbf{$n_t$:} number of time steps
	\item \textbf{$r$:} radius of the rocket
	\item \textbf{$P_{max}$:} maximum internal pressure
	\end{itemize}
	
	\subsection{Variables}
	\begin{itemize}
		\item \textbf{$x_{i,t}$:} x-dimension of the control point i at time t
		\item \textbf{$y_{i,t}$:} y-dimension of the control point i at time t
		\item \textbf{$n_{x_{i,t}}$:} normal vector of edge to the LHS of node i at time t
		\item \textbf{$n_{y_{i,t}}$:} normal vector of edge to the LHS of node i at time t
	\end{itemize}
	
	\subsection{Constraints}
	
	\begin{equation}
	\begin{aligned}
	& \min &&\sum_{i=1}^{n_t} (P_{i} - P_{des})^2\\
	& \text{ s.t.} && r^2 \geq x_{i,t}^2 + y_{i,t}^2, \forall i, t \\
	& && l_{i,t}^2 = (x_{i,t} - x_{i-1,t})^2 + (y_{i,t} - y_{i-1,t})^2 \\
	& && n_{x_{i,t}} = \frac{-(y_{i,t}-y_{i-1,t})}{l_{i-1,t}} \\
	& && n_{y_{i,t}} = \frac{x_{i,t}-x_{i-1,t}}{l_{i-1,t}} \\
	\end{aligned}
	\end{equation}
	
	\section{Modifying Constraints}

	\subsection{Relaxing the Diameter Constraint}
		
	\subsection{Non GP compatible constraints}
	The three non GP compatible constraints are:
	
	
	% --------------------------------------------------------------
	%     You don't have to mess with anything below this line.
	% --------------------------------------------------------------
	
\end{document}