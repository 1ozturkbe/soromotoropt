% --------------------------------------------------------------
% This is all preamble stuff that you don't have to worry about.
% Head down to where it says "Start here"
% --------------------------------------------------------------

\documentclass[12pt]{article}

\usepackage[margin=1in]{geometry} 
\usepackage{amsmath,amsthm,amssymb}
\usepackage{comment}
\usepackage{graphicx}

\newcommand{\N}{\mathbb{N}}
\newcommand{\Z}{\mathbb{Z}}

\newenvironment{theorem}[2][Theorem]{\begin{trivlist}
		\item[\hskip \labelsep {\bfseries #1}\hskip \labelsep {\bfseries #2.}]}{\end{trivlist}}
\newenvironment{lemma}[2][Lemma]{\begin{trivlist}
		\item[\hskip \labelsep {\bfseries #1}\hskip \labelsep {\bfseries #2.}]}{\end{trivlist}}
\newenvironment{exercise}[2][Exercise]{\begin{trivlist}
		\item[\hskip \labelsep {\bfseries #1}\hskip \labelsep {\bfseries #2.}]}{\end{trivlist}}
\newenvironment{reflection}[2][Reflection]{\begin{trivlist}
		\item[\hskip \labelsep {\bfseries #1}\hskip \labelsep {\bfseries #2.}]}{\end{trivlist}}
\newenvironment{proposition}[2][Proposition]{\begin{trivlist}
		\item[\hskip \labelsep {\bfseries #1}\hskip \labelsep {\bfseries #2.}]}{\end{trivlist}}
\newenvironment{corollary}[2][Corollary]{\begin{trivlist}
		\item[\hskip \labelsep {\bfseries #1}\hskip \labelsep {\bfseries #2.}]}{\end{trivlist}}

\begin{document}
	
	% --------------------------------------------------------------
	%                         Start here
	% --------------------------------------------------------------
	
	%\renewcommand{\qedsymbol}{\filledbox}
	
	\title{Solid Rocket Motor Optimization for Additive Manufacturing}
	\author{Berk Ozturk}
	
	\maketitle
	
	\section{Motivation}
	
There are various methods of performing shape optimization, which is inherently an infinite-dimensional problem. One approach is to optimize directly for the continuous quantity of interest, a function, which is often done iteratively. Starting with an initial guess, partial differential equations are evaluated over the domain of interest, and then the shape is perturbed until some optimality condition is reached. 

Another approach is to discretize the space, which has other difficulties. [See \textit{Pros and Cons of Airfoil Optimization} by Mark Drela]. However, this method can be used to great effect when the thing being designed is uncontinuous, such as things meant to be additively manufactured. 

These is very little literature, if any, on rocket conceptual design, for several reasons. (1) The physics governing internal reactive flow is extremely complex, and there are no open-source tools that can efficiently simulate such a scenario. (2) There are few entities that design and build solid rocket motors, and since they have large amounts of resources and time and little competition there has been no impetus improve solid rocket design methods. (3) Solid rocket motors are relatively limited in capability because their burn rate is almost completely uncontrollable other than by properly designing the internal geometry of the rocket or potentially actuating the throat of the nozzle. For this reason liquid propellant rockets have been preferred for many applications. (4) Rocket design is often proprietary due to arms regulations. 

As a result, it is worthwhile from both a theoretical and practical standpoint to come up with a framework to perform the conceptual design of solid rocket motors that does not rely on prior experience and vast amounts of time and capital resources. 
	
	\section{Concept}
	
	To tackle this problem, we will first make a series of approximations to make it more tractable. 
	
	\subsection{Assumptions}
	\begin{itemize}
		\item \textbf{The internal flow is steady and isentropic:} This will allow us to make approximations for the static pressure inside the rocket, which is related to the burn rate.
		\item \textbf{Pressure dependence on the burn rate is defined by the Saint-Robert's law:} $r = a P_c^n$
		\item \textbf{Constant radial burn rate throughout the chamber:} This allows us to treat the problem as a 2D problem. This approximation is justifiable since there is a relative pressure drop in the bore as the fluid accelerates through it. The greater rate of erosive burning near the exit port is offset by the lower pressure near the port. 
		\item \textbf{The grain is not temperature sensitive:} This is definitely not a correct assumption, but since this is a proof of concept we can assume this in the beginning. 
	\end{itemize}
	
	
	Starting with $n_{ctrl}$ control points evolving over $n_t$ time steps, we can devise a method to appropriately mesh the 2D slice of the rocket. 
	The hope is that the problem can be posed as a difference-of-convex optimization problem, and especially a mixed integer convex problem through piecewise linearization and the use of binary variables. 
	The pipe dream is to be able to perform this optimization in 3D with good first-order methods to accommodate for erosive burning. 
	
	\subsection{Alternative ways to think about the burn rate}
	We can also think about the burn rate being faster in regions that are more exposed to the flame (thinking about the dot product of the two vectors originating from the point), and then assigning a binary value on whether or not the resulting vector is within or outside of the burning surface depending on the cross product of the two vectors. However, this is more easily said than done, since both of these vector operations are non-convex. 
		
	
	\section{Preliminary Optimization Problem}
	
	To test whether or not the method of vector translations can be solved
	in a mixed integer convex form, we formulate the following problem 
	
	\subsection{Parameters}
	\begin{itemize}
	\item \textbf{$n_{ctrl}$:} number of control points
	\item \textbf{$n_t$:} number of time steps
	\item \textbf{$r$:} radius of the rocket
	\item \textbf{$R$:} regression rate of fuel
	\item \textbf{$C_{profile}$:} burn surface length desired (vector of $n_t$)
	\end{itemize}
	
	\subsection{Variables}
	\begin{itemize}
		\item \textbf{$x_{i,t}$:} x-dimension of the control point i at time t
		\item \textbf{$\Delta x_{i,t}$:} vector from point $x_{i-1,t}$ to $x_{i,t}$.
		\item \textbf{$y_{i,t}$:} y-dimension of the control point i at time t
		\item \textbf{$\Delta y_{i,t}$:} vector from point $y_{i-1,t}$ to $y_{i,t}$.
		\item \textbf{$l_{i,t}$:} length of vector from $(x,y)_{i-1,t}$ to $(x,y)_{i,t}$  
		\item \textbf{$C_{t}$:} the length of the burn surface at time t
		\item \textbf{$n_{x_{i,t}}$:} normal vector of edge to the LHS of node i at time t
		\item \textbf{$n_{y_{i,t}}$:} normal vector of edge to the LHS of node i at time t
	\end{itemize}
	
	\subsection{Untransformed constraints}
	
	The raw formulation, without any regard for linearity or convexity
	is as follows:
	
	\begin{align}
	& \min &&\sum_{t=1}^{n_t} (C_{t} - C_{profile})^2 \\
	& \text{ s.t.} && r^2 \geq x_{i,t}^2 + y_{i,t}^2, \forall i, t \label{eq1} \\
	& && \Delta x_{i,t} = x_{i,t} - x_{i-1,t} \label{eq2}\\
	& && \Delta y_{i,t} = y_{i,t} - y_{i-1,t} \label{eq3} \\
	& && l_{i,t}^2 = \Delta x_{i,t}^2 + \Delta y_{i,t}^2 \label{eq4} \\
	& && C_{t} = \sum\limits_{i=1}^{n_{ctrl}} l_{i,t} \label{eq5} \\
	& && n_{x_{i,t}} = \frac{-\Delta y_{i,t}}{l_{i,t}} \label{eq6} \\
	& && n_{y_{i,t}} = \frac{\Delta x_{i,t}}{l_{i,t}} \label{eq7} \\
	& && x_{i,t+1} = x_{i,t} + \frac{1}{2} R (n_{x_{i,t-1}}+ n_{x_{i+1,t-1}}) \label{eq8} \\ 
	& && y_{i,t+1} = y_{i,t} + \frac{1}{2} R (n_{y_{i,t-1}}+ n_{y_{i+1,t-1}}) \label{eq9} \\
	\end{align}
	
	\section{Dealing with nonlinearities and nonconvexities}
	\begin{itemize}
		\item \textbf{Objective:} Nonlinear but convex. Can leave be. 
		\item \textbf{Equation~\ref{eq1}:} Nonlinear but convex. Can leave be. 
		\item \textbf{Equation~\ref{eq2}:} Linear. 
		\item \textbf{Equation~\ref{eq3}:} Linear.
		\item \textbf{Equation~\ref{eq4}:} Linear.
		\item \textbf{Equation~\ref{eq5}:} Since this is an equality it is non-convex. We also want this to be tight. So we insert two auxiliary variables, and do a piecewise linearization of the quadratic terms.
%	\begin{align}
%	l_{i,t}^2 = \Delta x_{i,t}^2 + \Delta y_{i,t}^2
%	\end{align}
		\item \textbf{Equation~\ref{eq6}:} Linear. 
		\item \textbf{Equation~\ref{eq7}:} Nonlinear, and convexity depends on sign of $\Delta y$. 
		\item \textbf{Equation~\ref{eq8}:} Nonlinear, and convexity depends on sign of $\Delta x$.
		\item \textbf{Equation~\ref{eq9}:} Linear. 

	\end{itemize}
	
	\section{Modifying Constraints}

	\subsection{Relaxing the Diameter Constraint}
		
	\subsection{Non GP compatible constraints}
	The three non GP compatible constraints are:
	
	
	% --------------------------------------------------------------
	%     You don't have to mess with anything below this line.
	% --------------------------------------------------------------
	
\end{document}